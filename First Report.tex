\documentclass[14]{article}
\usepackage[utf8]{inputenc}
%\usepackage[swedish]{babel}
\usepackage{todonotes}
\usepackage{textcomp}
\usepackage{verbatim}
%\usepackage{hyperref}
\usepackage{url}
\usepackage{dirtytalk}
\usepackage[autostyle]{csquotes}
%\usepackage[]{natbib}
%\usepackage{apacite}
\usepackage{graphicx}
\usepackage{amsmath}
\usepackage{float}
\usepackage{tikz}
\usepackage{gensymb}
\usepackage{amssymb}
\usepackage[explicit]{titlesec}
\restylefloat*{figure}
\usepackage{listings}
\usepackage{color}
\usepackage[T1]{fontenc}
\usepackage{subcaption}
\usepackage{enumitem}
\usepackage{cleveref}
\usepackage{graphics}
\usepackage{multirow}
\usepackage{pythonhighlight}
\usepackage[colorlinks = true,
            linkcolor = blue,
            urlcolor  = blue,
            citecolor = blue,
            anchorcolor = blue]{hyperref}

\captionsetup[subfigure]{subrefformat=simple,labelformat=simple}
\renewcommand\thesubfigure{(\alph{subfigure})}
%Formal tables
\usepackage{booktabs}
\usepackage{eso-pic}								% Create cover page background
\usepackage{algorithmic}
\usepackage{algorithm}
\newcommand{\backgroundpic}[3]{
	\put(#1,#2){
	\parbox[b][\paperheight]{\paperwidth}{
	\centering
	\includegraphics[width=\paperwidth-4.5cm,height=\paperheight,keepaspectratio]{#3}}}}


\definecolor{dkgreen}{rgb}{0,0.6,0}
\definecolor{gray}{rgb}{0.5,0.5,0.5}
\definecolor{mauve}{rgb}{0.58,0,0.82}

\lstset{frame=tb,
  language=Java,
  aboveskip=3mm,
  belowskip=3mm,
  showstringspaces=false,
  columns=flexible,
  basicstyle={\small\ttfamily},
  numbers=none,
  numberstyle=\tiny\color{gray},
  keywordstyle=\color{blue},
  commentstyle=\color{dkgreen},
  stringstyle=\color{mauve},
  breaklines=true,
  breakatwhitespace=true,
  tabsize=3
}

\newcommand{\changed}[1]{\textcolor{red}{#1}}
\newcommand{\TODO}[1]{\todo[inline]{{\footnotesize #1}}}


\usepackage[position=b]{subcaption}
\begin{document}

\iffalse
% Agile Development Processes Project Report  
  Team X
% Team Member A; Team Member B
%

This is the project report template.
Fill it out over the course of the project.

# Project Description
Include the project description from the assignment,
your chosen target platform,
programming language,
and a name for the software.

Link to your:

- git repository
- issue tracker
- continuous integration builds

# Sprint 1 Log
Per sprint, fill out one sprint log section and its subsections.

## Commitment
List the features/stories that the team committed to finish during the sprint.

## Work Done
Log what was accomplished, and how.
Please report on all activities; for example, in addition to coding, planning and design discussion.

Feature | Time estimated | Time spent per team member
--------|----------------|--------
*Name and ID of each feature* | *X hours* | *Member A: Y hours, Member B: Z hours*

## Reflections
Reflect on how the work worked.
This data will form the basis for your final reflection.
As the postmortem will be a writeup, it's fine to use shorthand notes, bullet list, and similar.
Keep within 1000-1500 words.

Discuss any deviations from the sprint commitment.

Reflect on the agile practice practiced:

- Did your experience correspond to or contradict with what literature claims?

    - Analysis of why. Mostly interesting if something unexpected happens, but even
      if everything runs according to plan, reflecting on the underlying mechanisms
      can be interesting.

- How did the practices interact?
  Did they complement or counteract each other?

- How efficient were the practices, given the time they took to use?


# Sprint X Log
*As for the previous sprints.*


# Postmortem
Once the project is finished, summarize your experiences.
The postmortem part shall be 2000-3000 words long.

Considering the following:

- With regards to the agile practices, reflect on the combined experience from all sprints.

- Which practices had the most impact on the software developed?
  Think of both positives and negatives.

- What would you do differently in a future but similar project?


# Project outcome
Document the project, for example using screenshots.
\fi

\section{Project Description}
SpaceY is a visionary company that launches rocket shaped objects into space. We often have to type in a lot of rocket calculations into spreadsheets which requires us to memorise them. Therefore we are looking for someone to create an app which stores rocketry equations and lets us type in the required values so we can validate the spreadsheets. This should be a very simple application with a strong focus on the UX.
\\
\href{https://github.com/Owlinated/adp_project}{Project on github}

\section{}

\end{document}